\documentclass[a4paper,12pt,uplatex]{jsreport}

\usepackage{bm}
\usepackage[dvipdfmx]{graphicx}
\usepackage{ascmac}

\title{TITLE}
\author{AAA\\BBB}
\date{\today}

\begin{document}
\maketitle
\tableofcontents

\chapter{はじめに}
\section{DDD}

アイテマイズ
\begin{itemize}
	\item 第1章: 大まかな社会的背景や論文構成を述べる。
	\item 第2章: 本研究で扱うトラウマや治療法の説明を行う。
	\item 第3章: 先行研究の成果と問題点を述べる。
	\item 第4章: 問題点を整理し、解決手法と実装方法を述べる。
	\item 第5章: 実際に作成したアプリケーションの、機能の解説と実験と結果を述べる。
	\item 第6章: 本研究のまとめと今後の課題について述べる。
\end{itemize}


次の図\ref{FIG:p1}に示す。
\begin{figure}[h]
	\fontsize{12pt}{0cm}\selectfont
	\centering
	\renewcommand{\figurename}{図}
	\includegraphics[height=5cm]{img.jpg}
	\caption{パルサー}
	\label{FIG:p1}
\end{figure}

表
\begin{table}[htbp]
    \caption{表タイトル}
    \centering
    \begin{tabular}{|l||r|r|}
        \hline
        モード名 & 取得間隔 & 1秒間に取得できるデータ数\\
        \hline
        \hline
        AAA & BBB & CCC\\
        \hline
        AAA & BBB & CCC\\
        \hline
        AAA & BBB & CCC\\
        \hline
        AAA & BBB & CCC\\
        \hline
    \end{tabular}
    \label{TB:1}
\end{table}\\


次の図\ref{FIG:app2}, \ref{FIG:app3}を示す。\\
\begin{figure}[htbp]
	\begin{minipage}[b]{0.45\linewidth}
		\centering
		\includegraphics[keepaspectratio, scale=0.2]{xxx.jpg}
		\caption{XXX}
		\label{FIG:yyy}
	\end{minipage}
	\begin{minipage}[b]{0.45\linewidth}
		\centering
		\includegraphics[keepaspectratio, scale=0.2]{yyy.jpg}
		\caption{YYY}
		\label{FIG:xxx}
	\end{minipage}
\end{figure}\\

\end{document}
